
\documentclass[a4paper]{article}

\usepackage[utf8]{inputenc}

\usepackage{geometry}

\geometry{legalpaper, margin=1in}

\usepackage[T1]{fontenc}

\usepackage{algorithm}

\newcommand\tab[1][1cm]{\hspace*{#1}}

\usepackage{graphicx}



\title{SDD : TP 2}

\author{Mathieu Boutin - Jérémy Morceaux}

\begin{document}
\maketitle
\section{Présentation générale}
- Les fichiers sources se trouvent dans le dossier \textbf{src}.Les fonctions relatives aux Piles sont dans le fichier \textbf{ZZ\_Pile.c} et celles des Files sont dans le fichier \textbf{ZZ\_Pile.c}.Les entêtes sont dans le fichier \textbf{include}.

\section{Détail de chaque fonction}

\subsection{initStack}

\begin{algorithm}

Principe : initStack
\\
\\
\tab On initialise un pointeur pointant vers la tête de la pile 
\\
\tab Si le pointeur est non Null alors:
\\
\tab \tab On alloue une liste contiguë.
\\
\tab \tab  Si l'allocation s'est bien déroulée alors:
\\
\tab \tab \tab On affecte des valeurs à la pile.
\\
\tab \tab Sinon:
\\ 
\tab \tab \tab  On libère la pile.
\\
\\
\tab On retourne l'adresse de la tête de la pile
\\



FIN

\end{algorithm}
\underline{Lexique :}

\begin{itemize}

\item Paramètre(s) de la fonction  

\begin{itemize}

\item size est la taille de la Pile.

\item errorCode est un pointeur sur un entier qui indique si la fonction s'est bien déroulée.

\end{itemize}

\item Variable(s) locale(s)

\begin{itemize}

\item p est un pointeur de la Pile.
\end{itemize}

\end{itemize}

\underline{Programme commenté :}
\subsection{freeStack}

\begin{algorithm}

Principe : freeStack
\\
\\
\tab Si la liste contiguë existe alors:
\\
\tab \tab On libère la liste contiguë.
\\
\tab  Sinon:
\\ 
\tab \tab  On libère la pile.
\\
\tab On affecter au pointeur la valeur NULL.
\\


FIN
\end{algorithm}
\underline{Lexique :}

\begin{itemize}

\item Paramètre(s) de la fonction  

\begin{itemize}

\item p est la tête fictive de la Pile.

\end{itemize}

\end{itemize}

\underline{Programme commenté :}

\newpage
\subsection{isEmpty}

\begin{algorithm}

Principe : isEmpty

\tab On test si la la pile est vide. 
\\



FIN
\end{algorithm}


\underline{Lexique :}

\begin{itemize}

\item Paramètre(s) de la fonction  

\begin{itemize}

\item p est le pointeur de tête fictive de la Pile.

\end{itemize}

\end{itemize}

\underline{Programme commenté :}
\subsection{push}

\begin{algorithm}

Principe : push
\\
\\
\tab Si la pile n'est pas pleine alors:
\\
\tab \tab On attribue la valeur v au premier block libre de la liste contiguë.
\\



FIN

\end{algorithm}


\underline{Lexique :}

\begin{itemize}

\item Paramètre(s) de la fonction  

\begin{itemize}

\item p est le pointeur de tête fictive de la Pile.

\item errorcode est un pointeur sur un entier qui indique si la fonction s'est bien déroulée.

\item v est la valeur que l'on veut mettre dans la Pile.

\end{itemize}


\end{itemize}

\underline{Programme commenté :}
\subsection{pop}

\begin{algorithm}

Principe : pop
\\
\\
\tab Si la pile n'est pas vide alors:
\\
\tab \tab On récupère le dernier élément de la pile et on le supprime.


FIN

\end{algorithm}


\underline{Lexique :}

\begin{itemize}

\item Paramètre(s) de la fonction  

\begin{itemize}

\item p est le pointeur de tête fictive de la Pile.

\item errorcode est un pointeur sur un entier qui indique si la fonction s'est bien déroulée.

\item v est la variable dans laquelle on va mettre l'élément que l'on dépile.

\end{itemize}
\end{itemize}
\underline{Programme commenté :}
\newpage
\subsection{initFile}

\begin{algorithm}

Principe : initFile
\\
\\
\tab On initialise le pointeur pointant vers la File.
\\
\tab Si le pointeur est non NULL alors:
\\
\tab \tab On affecte des valeurs à la File pointée par le pointeur.



FIN

\end{algorithm}


\underline{Lexique :}

\begin{itemize}

\item Paramètre(s) de la fonction  

\begin{itemize}

\item p0 est le pointeur de tête fictive de la File.

\item errorcode est un pointeur sur un entier qui indique si la fonction s'est bien déroulée.

\item size est la taille de la File.

\end{itemize}

\end{itemize}
\underline{Programme commenté :}
\subsection{push}

\begin{algorithm}

Principe : pop
\\
\\
\tab Si la File n'est pas pleine alors:
\\
\tab \tab On ajoute l'élément au block suivant le dernier block occupé de la liste contiguë.

FIN

\end{algorithm}


\underline{Lexique :}

\begin{itemize}

\item Paramètre(s) de la fonction  

\begin{itemize}

\item p0 est le pointeur de tête fictive de la File.

\item errorcode est un pointeur sur un entier qui indique si la fonction s'est bien déroulée.

\item element est l'élément que l'on veut insérer dans la File.

\end{itemize}
\end{itemize}
\underline{Programme commenté :}
\subsection{pop}

\begin{algorithm}

Principe : pop
\\
\\
\tab Si la File n'est pas vjde alors:
\\
\tab \tab On retire le premier élément de la File.
\\
\tab On retourne l'élément supprimé.

FIN

\end{algorithm}


\underline{Lexique :}

\begin{itemize}

\item Paramètre(s) de la fonction  

\begin{itemize}

\item p0 est le pointeur de tête fictive de la File.

\item errorcode est un pointeur sur un entier qui indique si la fonction s'est bien déroulée.


\end{itemize}
\end{itemize}
\underline{Programme commenté :}
\subsection{estVide}

\begin{algorithm}

Principe : estVide
\tab On test si la la File est vide. 
\\



FIN
\end{algorithm}


\underline{Lexique :}

\begin{itemize}

\item Paramètre(s) de la fonction  

\begin{itemize}

\item p0 est le pointeur de tête fictive de la File.

\end{itemize}

\end{itemize}

\underline{Programme commenté :}
\subsection{libererFile}

\begin{algorithm}

Principe : libererFile
\\
\\
\tab Si la liste contiguë n'est pas vide alors:
\\
\tab \tab On libère la liste contiguë.
\\
\tab  Sinon:
\\ 
\tab \tab  On libère la File.
\\
\tab On affecter au pointeur la valeur NULL.
\\


FIN
\end{algorithm}
\underline{Lexique :}

\begin{itemize}

\item Paramètre(s) de la fonction  

\begin{itemize}

\item p0 est la tête fictive de la Pile.

\end{itemize}

\end{itemize}

\underline{Programme commenté :}
\end{document}